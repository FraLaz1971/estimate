\documentclass[a4paper,11pt]{article}
\usepackage[utf8]{inputenc}
\usepackage{color}
\usepackage{graphics}
\usepackage{graphicx}
%opening
\title{Montecarlo Estimation Errors}
\author{Francesco Lazzarotto}
\begin{document}
\maketitle
\begin{abstract}
The Monte Carlo method provides approximate
solutions to a variety of mathematical problems by performing statistical sampling experiments on a computer. Among all numerical methods that rely on $n$-points evaluation in $m$-dimensional space to produce an approximate solution, the Monte Carlo method has absolute error of estimate that decreases as $n^{-1/2}$ whereas,
in the absence of exploitable special structure, all others
have errors that decrease as $n^{-1/m}$ at best.
\end{abstract}
\input{estimate.tex}
\end{document}
